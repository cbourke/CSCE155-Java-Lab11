\documentclass[12pt]{scrartcl}



\setlength{\parindent}{0pt}
\setlength{\parskip}{.25cm}

\usepackage{graphicx}

\usepackage{xcolor}

\definecolor{darkred}{rgb}{0.5,0,0}
\definecolor{darkgreen}{rgb}{0,0.5,0}
\usepackage{hyperref}
\hypersetup{
  letterpaper,
  colorlinks,
  linkcolor=red,
  citecolor=darkgreen,
  menucolor=darkred,
  urlcolor=blue,
  pdfpagemode=none,
  pdftitle={CS1 - Lab 11.0 - Java},
  pdfauthor={Christopher M. Bourke},
  pdfkeywords={}
}

\definecolor{MyDarkBlue}{rgb}{0,0.08,0.45}
\definecolor{MyDarkRed}{rgb}{0.45,0.08,0}
\definecolor{MyDarkGreen}{rgb}{0.08,0.45,0.08}

\definecolor{mintedBackground}{rgb}{0.95,0.95,0.95}
\definecolor{mintedInlineBackground}{rgb}{.90,.90,1}

%\usepackage{newfloat}
\usepackage[newfloat=true]{minted}
\setminted{mathescape,
               linenos,
               autogobble,
               frame=none,
               framesep=2mm,
               framerule=0.4pt,
               %label=foo,
               xleftmargin=2em,
               xrightmargin=0em,
               startinline=true,  %PHP only, allow it to omit the PHP Tags *** with this option, variables using dollar sign in comments are treated as latex math
               numbersep=10pt, %gap between line numbers and start of line
               style=default, %syntax highlighting style, default is "default"
               			    %gallery: http://help.farbox.com/pygments.html
			    	    %list available: pygmentize -L styles
               bgcolor=mintedBackground} %prevents breaking across pages
               
\setmintedinline{bgcolor={mintedBackground}}
\setminted[text]{bgcolor={mintedBackground},linenos=false,autogobble,xleftmargin=1em}
%\setminted[php]{bgcolor=mintedBackgroundPHP} %startinline=True}
\SetupFloatingEnvironment{listing}{name=Code Sample}
\SetupFloatingEnvironment{listing}{listname=List of Code Samples}

\title{CSCE 155 - Java}
\subtitle{Lab 11.0 - Objects}
\author{~}
\date{~}

\begin{document}

\maketitle

\section*{Prior to Lab}

Before attending this lab:
\begin{enumerate}
  \item Read and familiarize yourself with this handout.
  \item Review the following free textbook resources:
	\begin{itemize}
  	  \item \url{http://docs.oracle.com/javase/tutorial/java/javaOO/index.html}
	\end{itemize}
  \item For additional Information on some advanced topics:
  \begin{itemize}
    \item RSS: \url{http://en.wikipedia.org/wiki/RSS}
    \item HTTP Protocol: \url{http://en.wikipedia.org/wiki/HTTP}
  \end{itemize}
\end{enumerate}

\section*{Peer Programming Pair-Up}

To encourage collaboration and a team environment, labs will be
structured in a \emph{pair programming} setup.  At the start of
each lab, you will be randomly paired up with another student 
(conflicts such as absences will be dealt with by the lab instructor).
One of you will be designated the \emph{driver} and the other
the \emph{navigator}.  

The navigator will be responsible for reading the instructions and
telling the driver what to do next.  The driver will be in charge of the
keyboard and workstation.  Both driver and navigator are responsible
for suggesting fixes and solutions together.  Neither the navigator
nor the driver is ``in charge.''  Beyond your immediate pairing, you
are encouraged to help and interact and with other pairs in the lab.

Each week you should alternate: if you were a driver last week, 
be a navigator next, etc.  Resolve any issues (you were both drivers
last week) within your pair.  Ask the lab instructor to resolve issues
only when you cannot come to a consensus.  

Because of the peer programming setup of labs, it is absolutely 
essential that you complete any pre-lab activities and familiarize
yourself with the handouts prior to coming to lab.  Failure to do
so will negatively impact your ability to collaborate and work with 
others which may mean that you will not be able to complete the
lab.  

\section{Lab Objectives \& Topics}
At the end of this lab you should be familiar with the following
\begin{itemize}
  \item Be familiar with the concepts of encapsulation \& modularity
  \item Understand how to design, declare, and use Java classes
  \item Have some exposure to advanced topics such as sockets, 
	the HTTP protocol, and XML processing
\end{itemize}

\section{Background}

An RSS feed (RDF Site Summary or ``Really Simple Syndication'') 
is a format used to publish frequently updated works.  RSS enabled 
clients can subscribe to RSS feeds and update a user as to new or 
relevant news items.  RSS feeds are most commonly formatted using 
XML (Extensible Markup Language) that use XML tags to indicate 
what the data represents (the title of the article, a short description, 
etc.).  Clients ``read'' an RSS feed by making a connection to a server 
using the HyperText Transfer Protocol (HTTP).

For example, UNL has an RSS news feed available at \url{http://newsroom.unl.edu/releases/?format=xml} 
which serves XML data that looks something like the following:

\begin{minted}[breaklines]{xml}
<rss xmlns:media="http://search.yahoo.com/mrss/" xmlns:atom="http://www.w3.org/2005/Atom" version="2.0">
<channel>
  <title>UNL News Releases</title>
  <link>http://newsroom.unl.edu/releases/</link>
  <description>News from the University of Nebraska-Lincoln</description>
  <language>en-us</language>
  <copyright>Copyright 2012 University of Nebraska-Lincoln</copyright>
  <image>
    <title>UNL News Releases</title>
    <url>http://www.unl.edu/favicon.ico</url>
    <link>http://www.unl.edu/</link>
  </image>
  <item>
    <title>Guerrilla Girls on Tour perform 'Feminists are Funny' Monday at Sheldon</title>
    <link>http://newsroom.unl.edu/releases/2012/03/09/Guerrilla</link>
    <description>The Guerrilla Girls on Tour, an internationally acclaimed anonymous theater collective, will perform "Feminists are Funny" at the University of Nebraska-Lincoln's Sheldon Museum of Art, 12th and R streets, at 7 p.m. March 12. The 70-minute play is an...
    </description>
    <pubDate>Fri, 09 Mar 2012 02:00:00 -0600</pubDate>
  </item>
  ...
  </items>
</channel>
\end{minted}

\subsection*{Classes in Java}

An entity may be composed of several different pieces of data.  
A person for example may have a first and last name (strings), 
an age (integer), a birthdate (some date/time representation), 
etc.  It is much easier and more natural to group each of these 
pieces of data into one entity.  This is a concept known as 
\emph{encapsulation}--a mechanism by which data can be grouped 
together to define an entity.

The Java programming language provides a mechanism to 
achieve encapsulation using classes.  In Java, everything 
(except for primitive types) is a class.  Many classes are 
provided by the SDK and users can define their own classes 
to model entities.  Classes can have one or more data 
fields--variables which have a type and a name as well 
as member methods.  

Instances of Java classes can be instantiated using a 
\emph{constructor}.  A constructor is a special method which 
has the same name as the class and which can be accessed 
using the new operator.  You can define any number of 
constructors (that take different number and types of 
parameters).  If you do not provide a user-defined constructor, 
Java provides a default, no-argument constructor that can be 
used.

Once you have an instance, you can access member variables 
and methods using the dot operator.  For example:

\begin{minted}{java}
MyObject myInstance = new MyObject();
myInstance.aPublicIntegerVariable = 10;
myInstance.executeSomeMethod();
\end{minted}

It is generally bad practice (poor use of encapsulation) to define 
a class with public member variables.  Instead, an instance's 
member variable values should be accessed through accessor 
and mutator methods (getters and setters) which can be
conveniently generated for you by Eclipse.

\subsection*{RSS Client Background}

You have been provided with an incomplete RSS client written 
in Java.  The client works as follows: it uses Java's URL class to 
open a connection to the given URL and obtain its raw RSS XML 
data.  The server responds with a stream of data that the URL 
class reads into a buffer.  This data stream can, in general, be 
any type of data, but we're expecting an RSS feed--a stream of 
plain text XML-formatted data conforming to the RSS standard.  
The data is placed into a \mintinline{java}{Document} object which parses the XML 
and provides an interface to access Nodes (XML elements) and attributes.

\section{Activities}

Clone the project code for this lab from GitHub by using the following 
URL:\\ \url{https://github.com/cbourke/CSCE155-Java-Lab11}.

\subsection{A Student Class}

This activity will familiarize you with a completed program in which 
a Java class has been created to represent a student.  A couple of 
constructors have been provided as well as getters and setters that 
offer several ways to build a \mintinline{java}{Student} object.  The 
standard \mintinline{java}{toString()} method has also been implemented 
that creates a human-readable string representation of the object.  
The conversion from a string representing a birthdate to a Java 8 
\mintinline{java}{LocalDate} object is also handled internally (encapsulated 
in the object).


\subsubsection*{Instructions}

\begin{enumerate}
  \item Examine the syntax of the \mintinline{java}{Student} class and 
  	understand how it works.
  \item Change the values in the \mintinline{java}{main} method to your 
	name, NUID, and birth date.
  \item Compile and run the program.  Refer back to this program in 
	Activity 2 as needed.
\end{enumerate}
	
\subsection{Completing the RSS Client}

In this activity, you will complete the RSS Client that connects to a 
UNL RSS feed, processes the XML data and outputs the results 
to the standard output.  Most of the client has been completed for 
you.  You just need to complete the design and implementation of 
a Java class that models the essential parts of an RSS item.  Your 
structure will need to support an RSS item's title, link, description, 
and publication date.  

\subsubsection*{Instructions}

\begin{enumerate}
  \item Open and examine \mintinline{text}{RssReader.java} and \mintinline{text}{Rss.java} 
  \item In \mintinline{text}{RssReader.java}, set the \mintinline{java}{DEFAULT_URL} 
  	value to the URL for the RSS feed that you want to pull from 
	(or add and use your own URL if you prefer).
  \item Design and implement the RSS class, \mintinline{text}{Rss.java}
  \begin{itemize}
    \item Define the class's state (its member variables)
    \item Define the class's constructors
    \item Define the class's methods (getters, setters, \mintinline{java}{toString}, etc.) as necessary
  \end{itemize}
  \item Complete the functionality in the \mintinline{text}{RssReader.java} class as follows.
  \begin{itemize}
    \item Use your new \mintinline{java}{Rss} class in the \mintinline{java}{getNewsFeeds} method 
    \item In the main method, add appropriate code to print out the RSS feed in 
    	a readable manner (the formatting details are up to you).  
  \end{itemize}
\end{enumerate}

\section{Advanced Activity (Optional)}

Many RSS feeds include escaped HTML characters that start with an 
ampersand (for example: to display a less-than sign, \mintinline{html}{&lt;} 
is used.  To display the ampersand itself, \mintinline{html}{&amp;} is used).  
This is necessary to avoid interpreting these characters as part of the HTML 
markup.  However, since we are printing it in a human readable format, 
it would be better to reformat these characters as the literal characters 
that they represent.  Write additional code to do this.  A full list of HTML 
character encodings can be found here: \url{http://www.w3schools.com/tags/ref_entities.asp}

\end{document}
